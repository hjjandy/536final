\section{Evaluation} \label{sec:evaluation}

In this section, we will show the results we got from our experiments. Our experiments will try to 
answer the following questions:
\begin{itemize}
    \item How is the overall throughput differ under different topologies?
    \item How is the fairness differ under different topologies?
    \item What is the conclusion of the experiment results?
\end{itemize}

The whole section is organized as the follows:
\begin{itemize}
    \item Part 1, a brief introduction of our experiment environments.
    \item Part 2, experiment results and analysis.
    \item Part 3, our suggestions according to the results we have.
\end{itemize}

\subsection{Environment Setup} \label{subsec:environment}
In our experiments, we use Mininet virtual box provided by official website\cite{Mininet:official}. 
The virtual machine runs Ubuntu Linux Quantal Quetzal on one CPU and $1024$ MB main memory.
The host machine we use has Intel\textregistered Core\texttrademark i7-3770 CPU with $12$ GB main memory running on
64-bit Windows Server 2008 R2 Enterprise version.

The network delay is set as 10{\it ms} since we simulate the network in a company, 
which typically might be higher in real-world networks \cite{NetworkDelay:web}. 
The network packet loss rate is 2\%, which is an acceptable rate as mentioned in \cite{PacketLoss:wiki}.
And all network links have a bandwidth of 100Mbps. 

We simulate each server for 10 iterations, obtaining the average results for 
analysis. Each iteration lasts for 2 minutes.

\subsection{Results and Analysis} \label{subsec:result}
We performed our experiments as discussed above. And we collected data reported by Mininet.
The notation h{\it x} where {\it x} $\in [0, 6]$ used here are the same with the ones 
shown in Figure~\ref{fig:topo}. In each table, the first column is the name of the server.
And for each client, we collected the speed value. The value is an average value of ten
experiments. And the total value in the last column means the total speed of all clients, 
which is also the overall throughput of the server. Since we have only one server in all the 
topologies, this value could be used to evaluate the overall throughput of the whole network. 
Notice that when a node is working as a server, it is not a client, that is why we have bar 
notations in the tables.

Table~\ref{table:line} here shows the result of the {\it Line Topology}. As we have 
mentioned before, there are 4 places that we can place our server in total. Let us first 
focus on only one rwo in the table, which means we just consider when the server is in a fixed
place, and we have some very interesting observation on different clients. Intuitively, 
the longer the distance between the server and the client, the worse the performance is. 
This is true. As we can see in the table, for the same server, as the distance grows, 
the speed of corresponding client is dropping down. However, it is not a strict linear 
relationship. For example, when h0 is the server, the nearest client 
and the farest client have distance $1$ and $6$ respectively. But the nearest client is 
almost $7.5$ times faster. Thus, when a cliet is near the server it gains far more benifit
than we expected. This is becasue the link between the server and the nearest client, here
it is h1, is shared by all clients. Thus this link is the one with highest probability that 
congestion could happen. However, h1, the nearest client here could take the most advantage of
this link because of the distance. Thus the speed is not a linear relationship with the distance.
This is an useful conclusion, we will use this to guide us to choose topologies later.

Now let us consider different rows. It is clear that as the server moving to the center of 
the line, the overall througput keeps growing. This is because the average distance between
server and clients becomes smaller as the server moving to the center of the line. But as we 
can see, the increasing speed is slowing down until the server reaches the center, where it 
stops growing anymore.
\begin{table}
	\renewcommand{\arraystretch}{1.3}
	\caption{Line Topology}
	\label{table:line}
	\setlength\tabcolsep{4pt} % adjust the inter-column space
	\centering
	\begin{tabular}{|c||c|c|c|c|c|c|c||c|}
		\hline
		       & \multicolumn{7}{c||}{Client} &  \\ \hhline{|~||-------||~|}
		Server & h0 & h1 & h2 & h3 & h4 & h5 & h6 & Total\\
\hline\hline
h0 &     -    &  607.70  &  373.10  &  250.70  &  170.90  &  122.66  &  81.10  & 1606.16 \\
\hline
h1 &  607.50  &     -    &  610.70  &  371.20  &  245.10  &  165.22  &  113.98  & 2113.70 \\
\hline
h2 &  373.89  &  606.80  &     -    &  610.90  &  365.80  &  239.60  &  160.10  & 2357.09 \\
\hline
h3 &  243.00  &  376.40  &  611.20  &     -    &  606.70  &  371.20  &  246.20  & 2454.70 \\
\hline
	\end{tabular}
\end{table}

Table~\ref{table:star} shows the result of the {\it Star Topology}. This is a pretty
simple topology for it has only two kinds of nodes. When the center node servers as the server 
all the other nodes have almost the same speed as we can see in the table. And we also find that 
almost all client nodes have used all the bandwidth it has to communicate width the server.
Thus this is a pretty good topology if we would like to maximize our overall throughtput.

However, when the server is placed at one edge. The overall throughput is not very good. This is just as
we expected. Since it has a longer distance between other edge nodes and the server. Even though 
the edge node still achieves a very good performance, the overall throughput is almost 2/3 of the
center server mode.
\begin{table}
	\renewcommand{\arraystretch}{1.3}
	\caption{Star Topology}
	\label{table:star}
	\setlength\tabcolsep{4pt}
	\centering
	\begin{tabular}{|c||c|c|c|c|c|c|c||c|}
		\hline
		       & \multicolumn{7}{c||}{Client} &  \\ \hhline{|~||-------||~|}
		Server & h0 & h1 & h2 & h3 & h4 & h5 & h6 & Total\\
\hline\hline
h0 &     -    &  612.60  &  595.70  &  608.30  &  606.50  &  611.60  &  610.60  & 3645.30 \\
\hline
h1 &  603.50  &    -     &  369.60  &  368.30  &  368.60  &  369.60  &  366.80  & 2446.40 \\
\hline
	\end{tabular}
\end{table}

Table~\ref{table:tree} here shows the result of the the {\it Tree Topology}. There are 3
different places for us to compare. For each place, the results also meet our expectations.
The distance between the server and the client is the main reason why the speed is different.
What make this topoloty interesting is that the overall throughput is better when h1 is the 
server node instead of {\it h0}. We originally expected that {\it h0} will serve the best, because the 
average distance is smellest when {\it h0} is the server (see Table~\ref{table:distance}). 
But this is not true. We further analyzed
the result, and found that this is because the reason we have mentioned when we were analyzing
the {\it Line Topology}. The relationship between the distance and the speed is not a linear
relationship.

Let us go into details about this. Let us assume that all links has the same weight, $1$. Then,
we have $3$ types of nodes: $GN$, whose distance is $1$ and get good service from servre; $MN$,
whose distance is $2$ and get medium service from server, and $BN$, whose distance is $3$ and get
bad service from the server. When h0 is the server node, we have two $GN$ and four $MN$. When h1
is the server node, we have three $GN$, one $MN$ and two $BN$. Since the performance-distance 
relationship follows a convex shape curve, the overall throught of one $GN$ and two $BN$ is better
than three $MN$. This example shows that we need to consider the distance carefully.
\begin{table}
	\renewcommand{\arraystretch}{1.3}
	\caption{Tree Topology}
	\label{table:tree}
	\setlength\tabcolsep{4pt}
	\centering
	\begin{tabular}{|c||c|c|c|c|c|c|c||c|}
		\hline
		       & \multicolumn{7}{c||}{Client} &  \\ \hhline{|~||-------||~|}
		Server & h0 & h1 & h2 & h3 & h4 & h5 & h6 & Total\\
\hline\hline
h0 &    -     &  604.10  &  600.00  &  364.30  &  372.44  &  355.30  &  369.10  & 2665.24 \\
\hline
h1 &  609.80  &     -    &  371.20  &  607.30  &  625.10  &  250.22  &  255.40  & 2719.02 \\
\hline
h3 &  372.90  &  606.30  &  240.00  &     -    &  368.90  &  170.11  &  165.33  & 1923.54 \\
\hline
	\end{tabular}
\end{table}

\subsection{Suggestions} \label{subsec:suggestions}

