\section{Discussion} \label{sec:discussion}

In this section, we will discuss some limitations and possible causes in our work.

There are many types of network topologies as mentioned in \cite{NetworkTopo:wiki},
but we only handled 3 of them in this paper. Our underlying platform for simulation
is Mininet \cite{Mininet:official}, which does not provide any mechanisms to construct
the bus topology. Also, even if we can write code to build ring and mesh topology,
Mininet itself is not able to process them because of the existence of loop in these
topologies \cite{MininetFAQ:web}. Some researchers proposed POX, which can handle the
loop routing by setting routing table \cite{POX:web}. But POX requires manual work 
to perform the simulation, and does not provide a way for automatic simulation. 
Due to the time limit, we omit this part in this paper.

Even if we did not evaluate ring and mesh topology, we can infer the results based on 
theoretical analysis and our evaluation. The ring topology is constructed by connecting
{\it s0} and {\it s6} in line topology. Then every host in the topology would have same 
behaviors as {\it h4} in line topology due to the nature of symmetry. 
For the mesh topology, if we only consider the case with all-to-all connection, we can
expect all hosts have the same behaviors as {\it h0} in star topology. Every switch 
connects to all the others directly, which means the distance is only 3 for any pair
of hosts along the shortest path, same as the case for the hosts transmitting data to 
{\it h0} in star topology. 

The evaluation has some problems too. Ideally, the server is connected with 6 hosts in
our simulation and so the result should contain statistics for 6 clients. But sometimes
the data for certain clients is missing. Our manual check discovered that all such cases 
happen for the farthest clients. For instance, in line topology, if {\it h0} acts as the 
server, data for {\it h6} may be failed to report. This is also a reason for us to 
simulate multiple times to obtain the average values. 

There are two potential reasons for the data loss. First, Mininet itself has some problems 
to report the data to the user. We also tried to simulate an aggressive case in which all 
hosts as as both server and client and data transmission occurs between any pair of 
client and server simultaneously. The results show that Mininet fails to output certain lines
of results. Second, the data is lost during the transmission. The data loss only occurs for 
the farthest clients, which transmits data to the server through multiple other switches and 
share the network links with multiple other clients. 
The switches may drop the data packets in some cases, \eg flow control or congestion control.

We omitted some factors that can influence the data transmission, \eg distance. If the network
exists in the same building, distance can be simply ignored. But if the company has several 
buildings locating in different places, the distance factor may have to be considered as it might 
greatly impact the efficiency and the best location for placing the server will be different. 
And in real-world networks, the delay and packet loss rate might not be identical for all 
network links. These complex cases need to be further considered in future work. 