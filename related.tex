\section{Related Work} \label{sec:related}

Strowes presented a solution to measure the TCP round trip time (RTT) \cite{Strowes:2013},
the result of which can be used to set the network delay parameter in our future simulations. 
Strowes applies different network conditions and packet loss rates in his experiments to 
obtain desirable results. Though we measure different aspects of the network, the methodology
can be employed in future work. Maguire measured the network latencies using DTrace \cite{Maguire:2010}
and Schlossnagle also presented experiments to measure the latencies
for services using TCP protocol \cite{Schlossnagle:2013}. 
All these results are helpful to determine the network delay parameter in simulations 
as in our work.

To assist the researchers
measuring the network latency and bandwidth, Russinovich released a tool called PsPing
\cite{PsPing:tool}, which can be used to measure the the real-world network behaviors. 
PsPing can complete similar job as in our work in this paper if provided with the same  
networks. The only difference is that our work is simpler to implement and 
perform with Mininet as the underlying network platform. 