\section{Related Work} \label{sec:related}

Strowes presented a solution to measure the TCP round trip time (RTT) \cite{Strowes:2013},
the result of which can be used to set the network delay parameter in our future simulations. 
Strowes applies different network conditions and packet loss rates in his experiments to 
obtain desirable results. Though we measure different aspects of the network, the methodology
can be employed in future work. Maguire measured the network latencies using DTrace \cite{Maguire:2010}
and Schlossnagle also presented experiments to measure the latencies
for services using TCP protocol \cite{Schlossnagle:2013}. 
All these results are helpful to determine the network delay parameter in simulations 
as in our work.

Fonseca {\it etc.} proposed a Bayesian approach to detect the packet loss in
TCP with low false positives \cite{FonsecaCrovella:Infocom05}. Their work suggest
that a more general packet loss inference in TCP can largely improve the throughput.
The results can be applied in constructing the network in a company, with same 
scenario as in this paper. In that case, our simulation needs to be changed
to adapt the new packet loss detection algorithm because the packet loss rate 
stays the same but the TCP throughput increases. The new algorithm will not 
affect our result of best location for the server if all other parameters 
stay unchanged. Savage presented {\it sting} to measure the network loss of TCP
protocol \cite{Savage:1999}, the results of which can be used in simulations to determine the 
packet loss rate parameter. 

To assist the researchers
measuring the network latency and bandwidth, Russinovich released a tool called PsPing
\cite{PsPing:tool}, which can be used to measure the the real-world network behaviors. 
PsPing can complete similar job as in our work in this paper if provided with the same  
networks. The only difference is that our work is simpler to implement and 
perform with Mininet as the underlying network platform. 

Lehman did a similar work  \cite{Lehman:2001} as our simulation in this paper.
He measured the throughput of TCP on a real-world network. While his result is
useful, it does not helpful for a company to make a decision of placing the 
server in a network. Phanishayee {\it etc.} measured the TCP throughput collapse in cluster-based
storage system \cite{Phanishayee:2008}. This problem is also need to 
be considered when determining the bast location for the server in our
scenario, depending on the purpose of the network.